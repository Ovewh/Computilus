\section{Conclusion}
In this study we have evaluated Monte Carlo Integration (MCI) for solving
physical complex problems with many degrees of freedom. We calculated the ground state correlation energy between two electrons in a helium
atom, which we implemented using two different MCI approaches, brute force MCI
and importance sampling MCI. Then we also made two implementations using
Gaussian Quadrature, a very different method, to compare against MCI. 

The results are divided into three parts: Firstly we looked at the reduction of
standard deviation for MCI with importance sampling compared to the brute force
approach. We observed that importance sampling MCI yielded a reduction in
standard deviation of a factor 10 or 
more for samples sizes greater than \num{1e8}. Secondly we compared the
two different GQ implementations (GQ Legendre and GQ Laguerre) and how the error
evolved with increased CPU time. Showing that GQ implementations did not benefit
significantly from more CPU time. Laguerre GQ actually yielded the lowest error 
for N between 14-15. Finally we did a comparison of all of our four
implementations where we looked at the error versus CPU time. Where
importance sampling MCI more or less made a clean sweep of the other
implementations, beating the second best, brute force MCI by more a factor of
[NEED A NUMBER] for longer runs. 

Lastly we want to point out that the physical problem we chose in our study was
not really fair towards GQ because of it's complexity and for a less
complex problem GQ might significantly outperform MCI.
 
