\section{Discussion}
The result from our analysis clearly favours MCI, particularly when
given more CPU time. The GQ approach did not benefit significantly with
increased
CPU time after a certain point, the sweet spot for GQ Laguerre being 14 and 15
integration points. Regardless we would not disregard using GQ, since the
accuracy of the method is very much dependent on the problem being evaluated.
In comparison MCI would behave more or less the same regardless of the physical
problem, since the standard deviation would still follow \cref{eq: std}.

Still we think this experiment did provide some insight into the kind of
situations where
these two very different approaches would be best suited. For instance if the
physical problem has many degrees of freedom $\nu \ge 6 $, then MCI would most
definitely yield better result. On the other hand, if the problem is less complex,
then GQ might yield very satisfactory results with very little computational cost.

Another point to be made based on our results is the potential huge benefits
from parallelization depending on how many cpu cores at your disposal. Combining
this with current increase and availability of multicore processors, makes a
even stronger case for MCI.
Then there is also the future prospect of quantum
computers, where quantum MCI can yield a quadratic improvement in error with
the same sample size \parencite{doi:10.1098/rspa.2015.0301}.
