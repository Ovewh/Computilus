\section{Introduction}

Physical systems are often too complex to be solved analytically due to having
too many physical parameters (degrees of freedom) which can all vary
independently. To circumvent having to solve for each degree of
freedom, we can instead randomly sample the behaviour of each parameter
and take the average. This kind of statistical approach is known as Monte Carlo
Integration (MCI).

Our aim is to demonstrate MCI, which we believe also yields
a rewarding insight into probability and statistics.  Then we
will demonstrate how to improve the accuracy of MCI, by
using a technique called importance sampling. We will also briefly describe
Gaussian Quadrature (GQ) which we will compare against MCI. The
physical problem we will look at is the ground state
correlation energy between two electrons in a helium atom. We compute the
correlation energy both using GQ and MCI. To be clear this work
will not go into any quantum mechanical details and we will treat this purely
as a mathematical problem.
