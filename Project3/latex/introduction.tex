\section{Introduction}

Many problems in physics involves adding stuff together, usually represented by
computing the integral of some function. Anyone familiar with physics knows that
these integrals often does not lend themselves to nice analytical solutions. In
these cases we have to use a numerical approximation/numerical integration of the integral.  
There are many procedures or rules we can use to numerically approximate an
integral. 

When first introduced to numerical integration, one might see the trapezoidal
rule, the midpoint or the Simpson's rule. These are all examples of
Newton-Cotes integration formulas. While these kind of methods is easy to
intuitively understand (in case of the trapezoidal rule the area under the curve
is approximated by adding together many trapezoids), they are usually not the
best when it comes to accuracy and efficiency. This is especially true for
higher dimensional integrals where the Newton-Cotes quadrature becomes highly
inefficient. 

A better method for solving higher dimensional integrals is Monte Carlo
Integration (MCI) (named after Monaco's famous casino district) also know as
Monte Carlo sampling.  
Our aim is to demonstrate MCI, which we believe will yield
many rewarding insights into the nature of probability and statistics. Then we
will demonstrate how to improve the accuracy of Monte Carlo Integration, by
using a technique called importance sampling. We will evaluate
performance of MCI by computing the ground state correlation
energy between two electrons in a helium atom. Finally we will compare the
performance of MCI with Gaussian quadrature using Legendre and Laguerre
polynomials.   

