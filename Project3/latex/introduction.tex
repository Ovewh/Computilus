\section{Introduction}

Physical system are often too complex to be solved analytically. Due to the
number physical parameters (degrees of freedom) which can all vary
independently. To circumvent having to solve for each degree of
freedom, we can instead randomly sample the behaviour of each parameter
and then take the average. This kind of statistical approach is knows as Monte Carlo
Integration (MCI) named after Monaco's famous casino district. 
  
Our aim is to demonstrate MCI, which we believe also yields
a rewarding insight into the nature of probability and statistics.  Then we
will demonstrate how to improve the accuracy of MCI, by
using a technique called importance sampling. We will also briefly describe
Gaussian Quadrature (GQ), which we will use to compare against MCI. The
physical problem we will solve using both QG and MCI is the ground state
correlation energy between two electrons in a helium atom. To be clear this work
will not got into any quantum mechanical details and we will treat this purely
as a mathematical problem. We mainly choose the specific problem since it has a
analytic solution which we can compare our result with. 