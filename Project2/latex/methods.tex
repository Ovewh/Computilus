\section{Methods}


Beskriv:
original ligning og skalering.

Considering the one dimensional wave function
\begin{equation}
\gamma \frac{d^2 u(x)}{dx^2} = -F u(x)
\end{equation}
, with u(x) being the vertical displacement of a beam in the y direction.
x $\in [0, L]$, with L being the length of the beam. $\gamma$ is a material constant,
and F is a force towards the origin being applied at the right hand side of the beam.
We will use the boundary conditions u(0) = u(L) = 0, meaning the beam doesn't move
at the end points.


Since we want to solve this equation numerically we scale it.

Introducing $ \rho = \frac{x}{L} $ limits $\rho$ to $[0, 1]$ and scales the
equation to
$$\frac{d^2 u(\rho)}{d\rho^2} = - \frac{FL^2}{\gamma} u(\rho) = -\lambda u(\rho).$$

We discretize $\rho$ on a grid with N points, which
define the step length as $h = \frac{\rho_N - \rho_0}{N}$ with $\rho_N = 1$ and
$\rho_0 = 0$.

Using the three point formula for the second derivative $ u'' = \frac{u(\rho + h) - 2u(\rho) +
u(\rho -u))}{h^2} + O(h^2)$, the initial equation can be discretized as
$\frac{-u_{i+1} + 2u_i - u_{i-1}}{h^2} = \lambda u_i$
or, in matrix form
$ A \vec{u} = \lambda \vec{u}$
with A being a tridiagonal matrix with $a = \frac{-1}{h^2}$ on the upper and lower
diagonal, and $d = \frac{2}{h^2}$ on the diagonal. $\vec{u}^T = [u_1, u_2, ..., u_{n-1}]$

Our equation is now in the form of an eigenvalue problem, which has analytical
eigenvalues $\lambda_j = d + 2a\cos{(\frac{j\pi}{n+1})}$ for $j \in [1,2,...,n]$
where n is the size of A.

Mathematical intermezzo.


Jacobi´s rotation algorithm.


\subsection{Extending to quantum mechanics}
Adding potential.

Approximating infinity.


\subsection{Bisection method}
