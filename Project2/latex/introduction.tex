\section*{Introduction}

High-level programming languages have in the recent years become increasingly
popular in scientific programming. The high-level syntax renders the code very
readable,
allowing the code for instance to closely resemble mathematics. The nice syntax
combined with excellent and easy to use libraries
for plotting and mathematics makes Python or Matlab the
programming languages of choice for many scientist.
The nice high-level features does not come without penalty, for
instance run large loops in Python or Matlab is known to be notoriously slow.
Therefore in high performance computing languages low level languages like C++/C
or Fortran are still the standard. 

Fortunately there exist solutions which tries to offer both the readable syntax
and ease of use of a high-level language and performance of low level languages.
We will
look at three different solutions, the just in time (JIT) compiler for python
Numba, Cython which is a C based extension of python and Armadillo a
library for scientific computing a in c++ offering a more high level syntax. 

First we will give an short description of the Numba, Cython and Armadillo. 
Next we will describe our experimental setup, where we implement the Jacobi
method for finding eigenvalues using the aforementioned approaches and
benchmarking them against a standard python approach.    


\subsection*{Numba}
Numba is an open source just in time (JIT) compiler which translates a subset of
python code 
into fast machine code. Currently Numba works best with NumPy, functions
and loops. Enabling numba in a python script is very simple and is done by
simply typing in a @jit deceleration above the python function you want to speed
up. If the @jit(nopython = true) Numba will compile the entire python function
without involving python at all, resulting in maximum speed up. Still to gain
any increase in speed at all, the code needs to consist of 
numpy functions or loops.   

\subsection*{Cython}
Cython is an optimising static compiler for Cython, which is a superset of the
python programming language. The cython compiler translate Python into C code by
adding static type decelerations. Cython offers the readability of high-level
python code combined with speed low level C and C++. This makes Cython ideal for
creating fast modules which can speed up otherwise slow python code. One of the
great advantages with Cython when developing code compared to for instance c++
or C, is that since Cython is a superset of Python we can actually start with
our slow Python code and then optimize the code iteratively. Compare this
approach to translating an entire python program into c++/c, which will probably
lead to mistakes during translation and unnecessary time spent debugging.
  
\subsection*{Armadillo}
Armadillo is an open-source linear algebra library for c++. Armadillo aims to
efficient while at the same time providing an easy-to-use interface. Armadillo
has a large variety of different linear algebra function provided through
integration of the linear algebra package (LAPACK).  
