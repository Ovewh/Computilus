\section*{Conclusion} 
We have investigated the performance of Cython and Numba, tools to easily
optimise python code and compared them with Armadillo. 
From our result we have shown that it is possible to achieve low-level
performance with a high-level language. We where able to make our baseline
python Jacobi's method implementation around 500 times faster through only
slight modification and even beating our c++ Armadillo implementation. 

Cython and Numba are quite different. With Numba 
the @jit compiler does all the work for you, which means doing any further
optimization beyond exploits in the physical problem is difficult. 
Cython offers the user much more control (and more work), giving 
much more flexibility in terms of optimizations. This is something we think is
important to consider when making optimizations. If you have larger and more
complex program, then you might achieve the best result by using Cython or
Armadillo. If instead you only need to optimise a small part of the code then
Numba might be the best choice, offering great performance for little effort. 
