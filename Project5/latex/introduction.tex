\section{Introduction}
Rossby waves was first introduced by the Swedish meteorologist
Carl G. Rossby in 1939 \parencite{Rossby1939}. The waves are large, with typical scales ranging from
hundreds to thousands of kilometres. Rossby waves play a major role in dictating
the weather on daily and large time scales and are crucial for the meridional
transport of heat, moisture and momentum \parencite{midSynDyn}.
Rossby waves are present both in the atmosphere and the ocean and are maintained
due to the latitudinal variation in the Coriolis acceleration.

In this study we will examine Rossby waves in a simplified
atmosphere/ocean. We will assume that our atmosphere/ocean is barotropic, that
is the density depends only on pressure, which via the ideal gas law holds that
the temperature is constant on a isobaric surface, i.e. no stratification. We also assume the flow to be frictionless and
non-divergent, consequently neglecting vertical velocities. Under these
simplifications the quasi geostrophic vorticity equation is reduced to
\cref{eq:QG_vorticity}, where $\zeta$ is the relative vorticity and $f$ is the
Coriolis parameter, $f=2\Omega \sin(\theta)$, where $\Omega$ is
the rotation rate of the earth $\Omega = 2\pi/ \mathrm{day}$.
\begin{equation}\label{eq:QG_vorticity}
    \frac{D(\zeta + f)}{Dt} = 0
\end{equation}
Following \cref{eq:QG_vorticity} we have that the
change in absolute vorticity $(\zeta + f)$ following the flow is zero, in other
words the absolute vorticity is conserved. When you have the vorticity in terms
of a conserved quantity it is commonly called potential vorticity (PV). If a fluid parcel is moving
poleward towards region of larger $f$, then there must be a compensating change
in $\zeta$ in order for the absolute vorticity to remain unchanged. This
balancing act between planetary and relative vorticity is the essential
mechanism behind
Rossby waves. We will derive the barotropic Rossby
wave equation (BRWE) from \cref{eq:QG_vorticity} in the subsequent section.

After we have developed our analytical framework, we will discretize the BRWE
and solve the BRWE in both one and two dimension, both with periodic and
closed boundary conditions. We will examine both the forward
difference and central difference scheme and consider their stability.
