\section{Introduction}
Rossby waves was first described by the Swedish meteorologist Carl G Rossby
\parencite{Rossby1939}. Rossby waves exist due to the latitudinal variation in
the Coriolis acceleration, where it is largest at the high latitude and zero at
equator. The varying Coriolis acceleration changes the vorticity of fluid
parcels as they moves to different latitudes. Rossby waves have scales of
hundreds to thousands of kilometres. Rossby 

We will study Rossby waves in a greatly simplified 
barotropic atmosphere/ocean. In this simplified barotropic world the density depends only
on pressure, which via the ideal gas law holds that the temperature is constant
on a isobaric surface. We also assume the flow to be frictionless and
non-divergent, consequently neglecting vertical velocities. Under these
simplifications the quasi geostrophic vorticity equation is reduced to
\cref{eq:QG_vorticity}, where $\zeta$ is the relative vorticity and $f$ is the
planetary vorticity. 
\begin{equation}\label{eq:QG_vorticity}
    \frac{D(\zeta + f)}{Dt} = 0
\end{equation}
From \cref{eq:QG_vorticity} we also have the conservation of absolute vorticity
($\zeta + f$) following the flow. If a fluid parcel is moving poleward
towards region of larger $f$ then, there must be a compensating change in
$\zeta$ for order to keep the absolute vorticity unchanged. This balancing act
between planetary and relative vorticity is the essential mechanism behind
Rossby waves, demonstrating that even in this simplified form the dynamical
essence of Rossby waves is retained. 

