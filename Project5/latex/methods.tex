\section{Theory}

Rossby waves are described in what known as the Quasi Geostrophic (QG)
framework. The primary scaling assumption in QG theory is that the Rossby number
is much less that one. The Rossby number which can be obtained by scaling the
momentum equation is defined as the following:
\cref{eq:Rnumber},  
\begin{equation}\label{eq:Rnumber}
    R = \frac{U}{fL}
\end{equation}
Where $U$ is characteristic magnitude of the horizontal velocity, $f \sim 10^{-4}$
is the Coriolis parameter and $L$ the characteristic horizontal length scale.
From the assumption about the Rossby number it is evident that the horizontal
length scales has to be large. A feature of large scale motion in the atmosphere
and ocean is that there tends to be a balance between the pressure gradient
force and Coriolis acceleration acting perpendicular to the velocity vector. 
\begin{equation}\label{eq:geostrophic}
    fu = -\frac{1}{\rho}\pdv{p}{y} \quad fv = \frac{1}{\rho}\pdv{p}{x}
\end{equation} 
The issue with a purely geostrophic framework \cref{eq:geostrophic} is that
there is no development of the motion. 

In QG-theory velocities is
separated into a geostrophic and ageostrophic component $\mathbf{v} = \mathbf
{v_g} +\mathbf{v_a}$. The geostrophic velocities are
non-divergent and can consequently be written in terms of a steam function.
It is the ageostrophic wind which are responsible for the
horizontal divergence. Even though the ageostrophic velocities
are small compared to the geostrophic, they are important since
they governs the rising and sinking in the atmosphere.

The quasi geostrophic potential vorticity equation \cref{eq:QG_vort} can be
derived from the shallow water equations \textbf{CITE Joe}, following the
previously discussed
assumptions. 

\begin{equation}\label{eq:QG_vort}
    \left(\pdv{}{t}+u_g \pdv{}{x} + v_g \pdv{}{y} \right)(\zeta_g + \beta y) = 
    f_0 \pdv{w}{z}
\end{equation}

\subsection{Analytic expression of phase velocities}

The vorticity equation
\begin{equation}\label{eq:gov_vort}
    \pdv{\zeta}{t} + \beta \pdv{\psi}{x} = 0
\end{equation}
The barotropic Rossby wave equation \cref{eq:barotropicRossby}

\begin{equation}\label{eq:barotropicRossby}
    \pd{t}{} \nabla_H^2 \psi + \beta \pdv{\psi}{x} = 0 
\end{equation}
We assume the following periodic solution \cref{eq:waveSolution} and 
insert this into the barotropic Rossby wave equation \cref{eq:barotropicRossby}.
\begin{equation}\label{eq:waveSolution}
    \psi = A\cos{(2n\pi x /L - \omega t)}
\end{equation}
Using the one dimensional vorticity $\zeta_x = \pdv{v}{x} = \pdv[2]{\psi}{x}$ 

\begin{equation}
    \pdv{}{t} \left(-A (2n\pi/L)^2  \cos{(2n\pi x/L - \omega t)}\right) - \beta 
     (2n\pi/L) A \sin{(2n x\pi/L - \omega t) = 0}
\end{equation}

\begin{equation}
    -(2n\pi/L)\omega \cancel{A\sin{(2n x\pi/L - \omega t)}} - \beta 
    \cancel{A\sin{(2 xn\pi/L - \omega t)}} = 0
\end{equation}

\begin{equation}
    (2n\pi / L) \omega = \beta
\end{equation}

\begin{equation}\label{eq:omega1}
    \omega = \frac{-\beta L}{2n\pi} 
\end{equation}

The phase speed of the Rossby wave can be derived by considering the phase of
the wave \cref{eq:phase}.
\begin{equation}\label{eq:phase}
    \theta = \frac{n \pi x}{L} - \omega t
\end{equation} 
Then peak of the wave would be at $\theta = 2\pi$, where $\psi = A\cos{(2\pi)}$.
In therms of the atmospheric motion this peak would correspond to would
correspond to high pressure point. To get the motion of this point we can solve
\cref{eq:phase} for x, $x=\theta L / 2n \pi + \omega t L / 2n \pi$ and then
obtaining the phase speed by taking the time derivative. 
\begin{equation}
    c = \dv{x}{t} = \frac{\omega L}{2n\pi}
\end{equation} 
Inserting the expression for $\omega$ obtained previously \cref{eq:omega1}, we
get the following expression for the phase speed.
\begin{equation}\label{eq:Phase-speed1}
    c = \frac{-\beta L^2}{(2n\pi)^2}
\end{equation}
Then following \cref{eq:Phase-speed1} the Rossby wave will propagate 
toward the west.  
Next we assume a more realistic wavelike solution of \cref{eq:barotropicRossby}
where the amplitude of the wave is also a function of $x$. We
require there to be no flow at the endpoints, which we can enforce with simple
Dirichlet conditions $\psi= 0$ at $x=0$ and $x=L$.
\begin{equation}\label{eq:waveSolution2}
    \psi = A(x)\cos{(kx-\omega t)} 
\end{equation}  
Inserting \label{eq:waveSolution2} into the barotropic Rossby wave equation
\cref{eq:barotropicRossby} and then differentiating we get the following
equation:

\begin{equation}\label{eq:step1}
    \begin{split}
    \sin{(kx-wt)\left(\omega(A''(x)-k^2A(x))- \beta A(x)\right)} \\
    + \cos{(kx - \omega t)}\left(\omega 2kA'(x) +\beta A'(x)\right) = 0
    \end{split}
\end{equation}
The terms in front of the cosine and sine has to be zero, which means that we
can split equation in to two parts. The equation for the sine and cosine terms 
respectively.  

\begin{equation}\label{eq:sine-terms}
    \omega A''(x) -\omega k^2 A(x) - \beta A(x) = 0
\end{equation}
\begin{equation}\label{eq:cos-terms}
    \omega 2 k A'(x) + \beta A'(x) = 0
\end{equation} 

Solving the equation for the cosine terms \cref{eq:cos-terms} with respect to
$\omega$:
\begin{equation}
    \omega = -\frac{\beta}{2k}
\end{equation}
Inserting the expression for omega into \cref{eq:sine-terms} and solving the ODE
give the following expression for $A(x)$
\begin{equation}\label{eq:A_x}
    A(x) = C_1 \cos{(kx) + C_2 \sin{(kx)}}
\end{equation}
Imposing the boundary condition at $x=0$ requiring $C_1$ to be zero ruling
out the cosine term in \cref{eq:A_x}. Imposing the boundary condition $\psi=0$
at $x=L$ yields the following expression for A(x).
\begin{equation}
    A(x) = \sin{\left(\frac{n\pi}{L} x \right)}
\end{equation}
Then final solution becomes:
\begin{equation}
    \psi = \sin{\left(\frac{n\pi}{L} x \right)} \cos{(kx - \omega t)}
\end{equation}    
The phase speed can be calculated in the same manner as previously and inserting
the previously obtained expressions for $k$, $k=\frac{n\pi}{L}$ and $\omega$,
$\omega = -\frac{\beta}{k}$.
\begin{equation}
    c = \dv{x}{t} = \frac{\omega}{k} = -\frac{\beta}{2k^2} = -\frac{\beta L^2}{2 n^2 \pi^2}
\end{equation}

\section{Finite element differention}