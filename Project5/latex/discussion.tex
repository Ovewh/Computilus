\section{Discussion}

Our numerical model managed to accurately reproduce the analytical value of the
phase speed for the barotropic Rossby waves with sinusoidal initial state.
Extending our model to two dimension did not give more insight compared to
 one dimension, due to the limitation with the $\beta$-plane approximation, by
which we assume the Coriolis parameter to vary linearly around a reference
latitude. Consequently as we move further from the reference latitude the error
we are making becomes larger.

Making implications on the real behaviour of observed Rossby waves based on our
model is difficult due to our very simplified model. In fact  by assuming the
flow to be barotropic, we neglect the main driver of motion in the
atmosphere/ocean, specifically the poleward temperature gradient. Still, in the
end the purpose of using simplified models is not to create an accurate
representation of reality, it should rather a be considered a tool for building
our physical intuition. This intuition is invaluable when adding complexity to
the model, because the simple models can work as a baseline for evaluating of
the effects of added complexity. Additions to our model that would be a
interesting to follow up on, would be for example to include a mean flow, that
would make our model more representative of Rossby waves in the atmosphere,
where the westerlies greatly influences the direction of the waves.
