\section{Discussion}
Should say something about the numerical schemes 

\textbf{Might be more suitable in a conclusion??}

Our numerical model managed to accurately reproduce the analytical phase speed
for the barotropic Rossby waves with sinusoidal initial state. However making
implications on the behaviour real 
observed Rossby waves are problematic due to our very simplified model. In fact
by assuming the flow to be barotropic, we neglect the main driver of
motion in the atmosphere/ocean, namely the temperature gradient from the tropics
towards the poles. Still, in the end the purpose of using simplified models is
not to create an accurate representation of reality, it should rather a be
considered a tool for building our physical intuition. An intuition which will
is invaluable when adding complexity to the model, because the simple model can
be a baseline for examining of the effects from adding complexity.
Additions to our model that would be a interesting follow up on, would be for
example to include a mean flow to allow more accurately represent Rossby waves
in the atmosphere, where the westerlies greatly influences the direction waves.  