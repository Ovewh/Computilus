\section*{Methods} 
Building upon the previously described concepts of numerical derivatives, we will now describe how to solve our differential
equation \cref{eq:2}  numerically by rewriting it as a set of linear equations.
\par
Explicitly, we will solve the differential equation:
\begin{equation*}
  -u''(x) = f(x), \quad x \in (0,1), \quad u(0)=u(1)=0 
\end{equation*} 
We will define the discrete approximation to $u(x)$ as $v_i$ with grid points
$x_i = ih$ in the range from $x_0 = 0$ to $x_{n +1} = 1$, and the step length is
defined as $h = 1/(n+1)$. The boundary conditions is $v_0 = 0$ and 
$v_{n+1} = 0$. The second derivate we approximate according to
\cref{eq:secondorder} and also introducing the shorthand notation we get \cref{eq:2ndordshort}.  
\begin{equation}\label{eq:2ndordshort}
  g_i = -\frac{v_{i-1}-2v_i + v_{i+1}}{h^2} \quad \mathrm{for} \; i = 1,2,3 \dots , n
\end{equation}  
To see how \cref{eq:2ndordshort} can be represented as matrix equation, we will
first multiply each side by $h^2$.
\begin{equation*}
  v_{i-1} -2v_{i} + v_{i+1} = g_i h^2, \quad  \tilde{g}_i = g_i h^2
\end{equation*} 
Next we represent the $v_i$'s and the $\tilde{g}_i$'s as a vectors,
\begin{equation*}
  \bvec{v} = \left[v_1, v_2, v_3 ,\dots, v_{n}\right], \quad 
  \bvec{\tilde{g}} = \left[\tilde{g}_1,, \tilde{g}_2, \tilde{g}_3,
  \dots, \tilde{g}_{n} \right]
\end{equation*}  
Then if we transpose our two vectors we only need to find the $n\times n$ matrix
$\bvec{A}$ and our matrix equation is complete. The matrix $\bvec{A}$ would in
our case look like this.
\begin{equation*}
  \bvec{A} = 
  \begin{bmatrix}
    -2 & 1 & 0 & 0 &\dots & 0 \\
     1 & -2 & 1 & 0 & \dots & 0 \\
     0 & 1 & -2 & 1 & \dots & 0  \\
     \dots & \dots & \dots & \dots & \dots & \dots \\
     0 & \dots & & -1 & 2 & -1 \\
     0 & \dots & & 0 & -1 & 2
  \end{bmatrix}
\end{equation*}
It is easy to verify that $\bvec{A} v = \bvec{\tilde{g}}$ would give us 
\cref{eq:2ndordshort} by doing the matrix multiplication. 

\subsection*{Thomas Algorithm}
