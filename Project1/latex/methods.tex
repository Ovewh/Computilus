\section*{Methods} 
Building upon the previously described concepts of numerical derivatives, we will now describe how to solve our differential
equation \cref{eq:2}  numerically by rewriting it as a set of linear equations.
\par
To be explicit, the differential equation we will solve is:
\begin{equation*}
  -u''(x) = f(x), \quad x \in (0,1), \quad u(0)=u(1)=0 
\end{equation*} 
The first step is to define the discrete approximation $v_i$ to $u(x)$, with
grid points 
$x_i = ih$ in the range from $x_0 = 0$ to $x_{n +1} = 1$. Choosing the step
length to be defined as $h = 1/(n+1)$. Including boundary conditions as $v_0 =
0$ and $v_{n+1} = 0$. The second derivate is approximated according to
\cref{eq:secondorder}, but rewritten in shorthand notation \cref{eq:2ndordshort}.  
\begin{equation}\label{eq:2ndordshort}
  f_i = -\frac{v_{i-1}-2v_i + v_{i+1}}{h^2} \quad \mathrm{for} \; i = 1,2,3 \dots , n
\end{equation}  
To see how \cref{eq:2ndordshort} can be represented as matrix equation, we will
first multiply each side by $h^2$.
\begin{equation*}
  v_{i-1} -2v_{i} + v_{i+1} = f_i h^2, \quad  g_i = f_i h^2
\end{equation*} 
Next we represent the $v_i$'s and the $g_i$'s as a vectors,
\begin{equation*}
  \bvec{v} = \left[v_1, v_2, v_3 ,\dots, v_{n}\right], \quad 
  \bvec{g} = \left[g_1, g_2, g_3,
  \dots, g_{n} \right]
\end{equation*}  
Then if we transpose our two vectors we only need to find the $n\times n$ matrix
$\bvec{A}$ and our matrix equation is complete. The matrix $\bvec{A}$ would in
our case looks like this.
\begin{equation*}
  \bvec{A} = 
  \begin{bmatrix}
    2 & -1 & 0 & 0 &\dots & 0 \\
    -1 & 2 & -1 & 0 & \dots & 0 \\
     0 & -1 & 2 & -1 & \dots & 0  \\
     \dots & \dots & \dots & \dots & \dots & \dots \\
     0 & \dots & & -1 & 2 & -1 \\
     0 & \dots & & 0 & -1 & 2
  \end{bmatrix}
\end{equation*}
It is easy to verify that $\bvec{A} v = \bvec{\tilde{g}}$ would give us 
\cref{eq:2ndordshort}, simply by doing the matrix multiplication. The matrix $\bvec{A}$ has some
particular nice features, primarily it a tridiagonal matrix which means that we
can use the efficient Thomas algorithm to solve our linear system of equation,
secondly it has constant values along the diagonals, which we'll exploit in our
specialized Toeplitz algorithm.    
\subsection*{Thomas Algorithm}

The Thomas Algorithm is an general algorithm for solving tridiagonal sets of
linear equations. The algorithm is quite straight forward to implement and 
requires two steps only.
We will demonstrate this algorithm with a $4 \times 4$ matrix, and 
our equation is $\bvec{T} \cdot \bvec{u} = \bvec{g}$, see \cref{eq:4_4matrix}.
\begin{equation}\label{eq:4_4matrix}
  \begin{bmatrix}
    d_1 & c_1 & 0 & 0 \\
    a_1 & d_2 & c_2 & 0 \\
    0 & a_2 & d_3 & c_3  \\
    0 & 0 & a_3 & d_4 
  \end{bmatrix} \begin{bmatrix}
    u_1 \\ u_2 \\ u_3 \\ u_4
  \end{bmatrix} = \begin{bmatrix}
    g_1 \\ g_2 \\ g_3 \\ g_4
  \end{bmatrix}
\end{equation} 
Then the first step is to rewrite our matrix $T$ as upper triangular matrix.
Starting with the first row we multiply by it by $\frac{a_1}{d_1}$ and subtract
it from the second row.
\begin{equation*}
  0 + \left(d_2 - \frac{c_1 a_1}{d_1}u_2 \right)+ c_2u_3 = g_2 -g_1 
  \frac{a_1}{d_1}  
\end{equation*}  
Next we define $\tilde{d}_2 = d_2 - a_1 c_1/d_1$, 
$\tilde{g}_2 =  g_2 - g_1 c_1/d_1$ and repeat the same proses for the third row.
One also might see the general pattern emerging:

\begin{equation}
  \tilde{d}_i = d_i- a_{i-1} c_{i-1}/d_{i-1}, \quad \tilde{g}_i =  g_i - g_{i-1} c_{i-1}/d_{i-1}
\end{equation}

After doing this forward sweep the matrix equation does now look like this,
\begin{equation*}
  \begin{bmatrix}
    d_1 & c_1 & 0 & 0 \\
    0 & \tilde{d}_2 & c_2 & 0 \\
    0 & 0 & \tilde{d}_3 & c_3  \\
    0 & 0 & 0 & \tilde{d}_4 
  \end{bmatrix} \begin{bmatrix}
    u_1 \\ u_2 \\ u_3 \\ u_4
  \end{bmatrix} = \begin{bmatrix}
    \tilde{g}_1 \\ \tilde{g}_2 \\ \tilde{g}_3 \\ \tilde{g}_4
  \end{bmatrix}
\end{equation*}
and now we can solve our equation with respect to $\bvec{u}$ by starting from
the bottom row. We can directly read of $u_4 = \tilde{g}_4 / \tilde{d}_4$.
Working from the bottom and up we get $u_3 = \left(\tilde{g}_3-c_3
u_4\right)/\tilde{d}_3$. Then we have the following general pattern
\begin{equation}
  u_i = \left(\tilde{g}_i - c_i u_{i+1}/\tilde{d}_i \right)
\end{equation}  
Our implementation of the Thomas algorithm, is available
through our github\footnote{https://github.com/Ovewh/Computilus/tree/master/Project1/src/linalg.py}.


To get an idea of how an algorithm performs, we can count
the number of floating point operations per second (FLOPS). When counting FLOPS
we only look at the mathematical operations and only count those inside our
for-loops. Counting the number of FLOPS for the Thomas algorithm we get $9N$
FLOPS, which is considerably less than the standard LU decomposition which has
on the order of $N^3$ FLOPS.   
Memory usage is another important consideration to make when choosing
algorithms. For instance with the Thomas algorithm we only need to store the
values along the tridiagonal, since all the other elements are zero. Then we only need tree arrays to store the entire matrix. If we
would instead use the general LU decomposition algorithm which requires an $N
\times N$ matrix, we would need to store the entire matrix in memory. For
instance if we had a $10^5 \times 10^5$ matrix, which means $10^10$ matrix
elements. To store this matrix in memory when
every matrix element is 8 bytes, would require on the order of $10^{11}$
bytes, about 100 gigabytes to store. An amount of memory clearly beyond any
ordinary laptop. 

\subsection*{Specialized Algorithm}
To create a specialized version of Thomas algorithm for our tridiagonal Toeplitz
matrix we will take 
advantage of the fact that elements along the diagonals are constants. This
means that we can pre-calculate the new diagonal elements $\tilde{d}$, reducing
the number of FLOPS from $9N$ to $4N$
