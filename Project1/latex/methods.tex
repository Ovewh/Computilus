\section*{Method}
\subsection*{Numerical differentiation}
%NOTE THIS MAY END UP IN INTRODUCTION  
we need to discretize our problem.
The first  step in discretizing any problem is let our input variable $x$ be a discrete variable $x_i \in \left[x_0, x_1, x_2, ..., x_n \right]$. The distance between each $x_i$ variable is controlled by the step size, $h = \frac{x_n -x_0}{n}$, this gives an expression for $x_i = x_0 + i\cdot h$ where $i = 1,2,3, ..., n$. Then if we look at the definition of the derivative \ref{eq:1} it is clear to see how this would work for the discrete case.    
\begin{equation}\label{eq:1}
    f'(x) = \lim_{x\to 0} \frac{f(x+h)-f(x)}{h}
\end{equation}\label{eq:1} 
Since we obviously can't on a computer let $h = 0$ the derivate will always be an approximation. A natural question to ask when making approximations is what is our error. The error can be derived if we do a taylor series expansion of the $f(x+h)$ term in around $x$.
\begin{equation}\label{eq:4}
    f(x+h) = f(x) + h'f(x) + \frac{h^2f''(x)}{2} + \frac{h^3f'''(x)}{6} + \dots    
\end{equation}   

If we next insert this taylor expansion into eq.(\ref{eq:4}) we get:
\begin{equation}
  f'(x) = f'(x) + \frac{h f''(x)}{2} + \frac{h^2f'''(x)}{6} + \dots
\end{equation} 
Then we see that our derivate includes $f'(x)$ and some terms proportional to $h, h^2, h^3 \dots $ and since $h$ is small the $h$ terms would dominate.   
\par
Inorder to solve eq(\ref{eq:2}) numerically, we need to discretize our problem. We also assume Dirichlet boundary conditions $u(0) = u(1) = 0$, that $x = \; \in (0,1)$ and that our equation is scaled to avoid dealing with physical units. \par The first step in discretizing any problem is let our input variable $x$ be a discrete variable $x_i \in \left[x_0, x_1, x_2, ..., x_n \right]$. The distance between each $x_i$ variable is controlled by the step size, $h = \frac{x_n -x_0}{n}$, this gives an expression for $x_i = x_0 + i\cdot h$ where $i = 1,2,3, ..., n$. 
\par A widely used method to calculate the derivate numerically is what is called the 3 point method, (equation (\ref{eq:3})), where $f_{i \pm 1}$ is introduced as a shorthand for $ f(x_i \pm h)$.   

\begin{equation}\label{eq:3}
  f'_i = \frac{f_{i+1} - f_{i-1}}{2h}
\end{equation}
The three point formula is derived by evaluating the derivate at both sides of a chosen point $x_i$ and then taking the average afterwards. 
The three point formula has an error of order of magnitude $O(h^2)$ compared to a two point method which has $O(h)$, while requiring the same number of floating point operations (flops). The reasoning behind the three point method is that you evaluate $f(x_i)$ at $x_{i-1}$ and $x_{i+1}$ 