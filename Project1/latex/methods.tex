\section*{Methods} 
Building upon the previously described concepts of numerical derivatives
discussed earlier, we will now describe how to solve our differential
equation \cref{eq:2}  numerically by rewriting it as a set of linear equations.
\par
Explicitly, we will solve the differential equation:
\begin{equation*}
  -u''(x) = f(x), \quad x \in (0,1), \quad u(0)=u(1)=0 
\end{equation*} 
We will define the discrete approximation to $u(x)$ as $v_i$ with grid points
$x_i = ih$ in the range from $x_0 = 0$ to $x_{n +1} = 1$, and the step length is
defined as $h = 1/(n+1)$. The boundary conditions is $v_0 = 0$ and 
$v_{n+1} = 0$. The second derivate we approximate according to
\cref{eq:secondorder},
\begin{equation}\label{eq:2ndordshort}
  g_i = -\frac{v_{i-1}-2v_i + v_{i+1}}{h^2} \quad \mathrm{for} \; i = 1,2,3 \dots , n
\end{equation} 
where \cref{eq:2ndordshort} is rewritten using shorthand notation. 
Our next step will involve showing how \cref{eq:2ndordshort} can be written as a
linear system of equations $\bvec{A} \bvec{v} = \bvec{g}$. 

\subsection*{Thomas Algorithm}
