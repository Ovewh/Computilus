\section*{Introduction}
There is no denying the influence the computer has had on science. 
Computing has rendered scientist able to unravel the deeper mysteries of nature.
Today for instance a scientist can explore the violent nature of a
hurricane while at the same time enjoying a cup of coffee. With current rate in
increase of computational performance, we are diminishing the computational
limit. 
Still, to do computational problem solving both efficiently and with high
precision requires both an deep understanding of the problem at hand, but also the inner workings of a computer. 

Our aim is to demonstrate the benefits of understanding the problem at hand and to apply that knowledge to solve the problem in an efficient
way. For our demonstration with we will look at a how to solve a second order
differential equation, specifically the general one dimmensional Poisson's
equation \cref{eq:1}. 
\begin{equation}\label{eq:2}
  f(x)= - \pd[^2u]{x^2}
\end{equation}

\subsection*{Numerical differentiation}
%NOTE THIS MAY END UP IN INTRODUCTION  
\par Computers operate in discrete steps, which means that variables are stored
as discrete variables. A discrete variable  defined over a particular
range, would 
have a step length $h$ between each value and can not represent any values in
between. This means that how well a discrete variable would approximate the
continuous variable depends on the size of the step length. The step length $h$
can either be set manually or it can be determined based on the start and end
point of our particular range, $h = \frac{x_n -x_0}{n}$. Where $n$ is the number
of points we choose to have in our range. 
\par
The simplest way to compute the derivate numerically is to use what is called
forward difference method eq.(\ref{eq:1}) or equivalently backward difference
method (eq.\ref{eq:5}). If we include the limit $\lim_{h\to 0}$ we obtain the
classic definition of the derivate. 
\begin{equation}\label{eq:1}
    f'(x) \approx \frac{f(x+h)-f(x)}{h}
\end{equation}

\begin{equation}\label{eq:5}
  f'(x) \approx \frac{f(x-h)+f(x)}{h}
\end{equation}
Since numerical differentiation always will give an approximation of the
derivate, we would like to quantify our error. The error can be derived 
if we do a taylor series expansion of the $f(x+h)$ term in around $x$.
\begin{equation}\label{eq:4}
    f(x+h) = f(x) + h'f(x) + \frac{h^2f''(x)}{2} + \frac{h^3f'''(x)}{6} + \dots    
\end{equation}   
If we next insert this taylor expansion into eq.(\ref{eq:4}) we get:
\begin{equation}
  f'(x) = f'(x) + \frac{h f''(x)}{2} + \frac{h^2f'''(x)}{6} + \dots
\end{equation} 
Our approximation of the derivate includes $f'(x)$ and some terms which are
proportional to $h, h^2, h^3 \dots $ and since $h$ is assumed to be small the
$h$ terms would dominate. The error is said to be of the order $h$. 
\par
To get a numerical scheme for the second derivate we would just take the
derivate of \cref{eq:1} except for a slight modification. Instead of looking at $f''(x) \approx \frac{f'(x+h)-f'(x)}{h}$ we would use 
$f''(x) \approx\frac{f'(x)-f'(x-h)}{h}$, which are equivalent to each other
 \cite{mathexh}. 
\begin{equation}
  f''(x) \approx \frac{f(x + h) - f(x) -f(x-h+h) +f(x-h)}{h^2}
\end{equation}
Then after a bit of a clean up we get an approximation for the second order
derivate (\cref{eq:secondorder}). 
\begin{equation}\label{eq:secondorder}
  f''(x) \approx \frac{f(x+h)-2f(x) + f(x-h)}{h^2}
\end{equation}
Then to quantify the error we proceed as for the first order derivate, by
expanding $f(x+h)$ and $f(x-h)$. 
\begin{equation}\label{eq:taylor_f_x-h}
  f(x-h) = f(x)- h f'(x) + \frac{h^2 f''(x)}{2} - \frac{h^3f'''(x)}{6} \dots 
\end{equation} 
Next we substitute the two taylor expansion \cref{eq:taylor_f_x-h} and 
\cref{eq:4} into the expression for second order derivate \cref{eq:secondorder}.
\begin{equation}
  f''(x) \approx f''(x) + \frac{h^2f^{(4)}(x)}{4!} + \frac{h^4 f^{(6)}(x)}{6!} + \dots
\end{equation}
Then we see that leading error term is for the second derivate is 
$\mathcal{O}(h^2)$.   
