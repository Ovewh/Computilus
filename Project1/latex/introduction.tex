\section*{Introduction}
Computing has had and still have an undeniable influence on science.
has allowed scientist to explore everything from the tiniest scale of an atom,to
tropical cyclones and galaxies.
Therefore understanding the inner workings behind a computer program is critical
in order to avoid unwanted errors. Errors which in the worst case can have
catastrophic consequences \cite{sleipner_failure}. \par Our aim is to
investigate some of the common errors one might face if one doesn't think when
developing code. To begin with we will look at a how to solve a second order
differential equation, specifically the general one dimmensional Poisson's
equation (\ref{eq:1}).
\begin{equation}\label{eq:2}
  f(x)= - \pd[^2u]{x^2}
\end{equation}

\subsection*{Numerical differentiation}
%NOTE THIS MAY END UP IN INTRODUCTION  
\par Computers operate in discrete steps, which means that variables are stored
as discrete variables. A discrete variable defined over a particular range would
have step length $h$ between each value and can not represent any values in
between. This means that how well a discrete variable would represent the
continuous variable depends on the size of the step length. The step length $h$
can either be set manually or it can be determined based on the start and end
point of our particular range, $h = \frac{x_n -x_0}{n}$. Where $n$ is the number
of points we choose to have in our range. The shorter the step length in the
discrete variable the better it will approximate the continuous variable.  
\par
The simplest way to compute the derivate numerically is to use what is called
forward difference method eq.(\ref{eq:1}) or equivalently backward difference
method (eq.\ref{eq:5}). If we include the limit as$\lim_{h\to 0}$ we obtain the
classic definition of the derivate. 
\begin{equation}\label{eq:1}
    f'(x) \approx \frac{f(x+h)-f(x)}{h}
\end{equation}

\begin{equation}\label{eq:5}
  f'(x) \approx \frac{f(x-h)+f(x)}{h}
\end{equation}
Since numerical differentiation always will give an approximation of the
derivate, we would like to quantify our error. The error can be derived 
if we do a taylor series expansion of the $f(x+h)$ term in around $x$.
\begin{equation}\label{eq:4}
    f(x+h) = f(x) + h'f(x) + \frac{h^2f''(x)}{2} + \frac{h^3f'''(x)}{6} + \dots    
\end{equation}   
If we next insert this taylor expansion into eq.(\ref{eq:4}) we get:
\begin{equation}
  f'(x) = f'(x) + \frac{h f''(x)}{2} + \frac{h^2f'''(x)}{6} + \dots
\end{equation} 
Our approximation of the derivate includes $f'(x)$ and some terms which are
proportional to $h, h^2, h^3 \dots $ and since $h$ is assumed to be small the
$h$ terms would dominate. The error is said to be of the order $h$. 
\par
To get a numerical scheme for the second derivate we would just take the
derivate of \cref{eq:1} except for a slight modification. Instead of looking at $f''(x) \approx \frac{f'(x+h)-f'(x)}{h}$ we would use 
$f''(x) \approx\frac{f'(x)-f'(x-h)}{h}$, which are equivalent to each other
see. \cite{mathexh}. 
\begin{equation}
  f''(x) \approx \frac{f(x + h) - f(x) -f(x-h+h) +f(x-h)}{h^2}
\end{equation}
Then after a bit of a clean up we get an approximation for the second order
derivate (\cref{eq:secondorder}). 
\begin{equation}\label{eq:secondorder}
  f''(x) \approx \frac{f(x+h)-2f(x) + f(x-h)}{h^2}
\end{equation}
Then to quantify the error we proceed as for the first order derivate, by
expanding $f(x+h)$ and $f(x-h)$. 
\begin{equation}\label{eq:taylor_f_x-h}
  f(x-h) = f(x)- h f'(x) + \frac{h^2 f''(x)}{2} - \frac{h^3f'''(x)}{6} \dots 
\end{equation} 
Next we substitute the two taylor expansion \cref{eq:taylor_f_x-h} and 
\cref{eq:4} into the expression for second order derivate \cref{eq:secondorder}.
\begin{equation}
  f''(x) \approx f''(x) + \frac{h^2f^{(4)}(x)}{4!} + \frac{h^4 f^{(6)}(x)}{6!} + \dots
\end{equation}
Then we see that leading error term is for the second derivate is 
$\mathcal{O}(h^2)$.   
