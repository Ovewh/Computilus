% REMEMBER TO SET LANGUAGE!

\documentclass[a4paper,12pt,english]{article}
\usepackage[utf8]{inputenc}
\usepackage[english]{babel}
% Standard stuff
\usepackage{bm,amsmath,graphicx,varioref,verbatim,amsfonts,geometry, mleftright, siunitx}
\usepackage[nottoc]{tocbibind}
\usepackage{csquotes}
% colors in text
\usepackage[usenames,dvipsnames,svgnames,table]{xcolor}
\usepackage{chemformula}
\usepackage{caption}
\usepackage{subcaption}
\usepackage{gensymb}
\usepackage{float}
\usepackage{eso-pic}
\usepackage{physics}
\usepackage{titlepic}
\usepackage[T1]{fontenc}
\usepackage{booktabs}
% Nicer default font (+ math font) than Computer Modern for most use cases
\usepackage{mathpazo}
%Cite stuff
\usepackage[backend=biber,style=nature]{biblatex}
%\usepackage[colorlinks]{hyperref}
\usepackage{url}
% Document formatting
\setlength{\parindent}{0mm}
\setlength{\parskip}{1.5mm}

%Kryss ut ting
\usepackage[makeroom]{cancel}

%Color scheme for listings
\usepackage{textcomp}
\definecolor{listinggray}{gray}{0.9}
\definecolor{lbcolor}{rgb}{0.9,0.9,0.9}

\newcommand{\uveci}{{\bm{\hat{\textnormal{\bfseries\i}}}}}
\newcommand{\uvecj}{{\bm{\hat{\textnormal{\bfseries\j}}}}}
\DeclareRobustCommand{\uvec}[1]{{%
  \ifcat\relax\noexpand#1%
    % it should be a Greek letter
    \bm{\hat{#1}}%
  \else
    \ifcsname uvec#1\endcsname
      \csname uvec#1\endcsname
    \else
      \bm{\hat{\mathbf{#1}}}%
     \fi
   \fi
}}
\newcommand{\bvec}[1]{\mathbf{#1}}
%Fixing some ln shit!
\newcommand{\lnb}[1]{%
  \ln\mleft(#1\mright)%
}
\newcommand\BackgroundIm{
\put(0,0){
\parbox[b][\paperheight]{\paperwidth}{%
\vfill
\centering
\includegraphics[height=\paperheight,width=\paperwidth]{FrontpageBlue.png}%
\vfill
}}}

\newcommand{\pd}[2][]{\frac{\partial#1}{\partial#2}}
%Listings configuration
\usepackage{listings}
%Hvis du bruker noe annet enn python, endre det her for å få riktig highlighting.
\lstset{
	backgroundcolor=\color{lbcolor},
	tabsize=4,
	rulecolor=,
	language=python,
        basicstyle=\scriptsize,
        upquote=true,
        aboveskip={1.5\baselineskip},
        columns=fixed,
	numbers=left,
        showstringspaces=false,
        extendedchars=true,
        breaklines=true,
        prebreak = \raisebox{0ex}[0ex][0ex]{\ensuremath{\hookleftarrow}},
        frame=single,
        showtabs=false,
        showspaces=false,
        showstringspaces=false,
        identifierstyle=\ttfamily,
        keywordstyle=\color[rgb]{0,0,1},
        commentstyle=\color[rgb]{0.133,0.545,0.133},
        stringstyle=\color[rgb]{0.627,0.126,0.941}
        }
        



\setlength{\parindent}{1em}

\usepackage{etoolbox}
\makeatletter
\patchcmd{\@maketitle}{\vskip 2em}{\vspace*{2.2cm}}{}{}
\makeatother

\title{Project 1 }
\author{Ove Haugvaldstad, Eirik Gallefoss}

\addbibresource{references.bib}
\begin{document}
\pagenumbering{gobble}
\maketitle

\section*{Abstract}
\newpage
\pagenumbering{arabic}
\section*{Introduction}
Computing has had and still have an undeniable influence on science. 
has allowed scientist to explore everything from the tiniest scale of an atom,to tropical cyclones and galaxies. 
Therefore understanding the inner workings behind a computer program is critical in order to avoid unwanted errors. Errors which in the worst case can have catastrophic consequences \cite{sleipner_failure}. \par Our aim is to investigate some of the common errors one might face if one doesn't think when developing code. To begin with we will look at a how to solve a second order differential equation, specifically the general one dimensional Poisson's equation (\ref{eq:1}). 
\begin{equation}\label{eq:1}
  f(x)= - \pd[^2u]{x^2}
\end{equation}

\section*{Method}
In order to solve eq(\ref{eq:1}) numerically, we need to discretize our problem. We also assume Dirichlet boundary conditions $u(0) = u(1) = 0$, that $x = \; \in (0,1)$ and that our equation is scaled to avoid dealing with physical units. \par The first step in discretizing any problem is let our input variable $x$ be a discrete variable $x_i \in \left[x_0, x_1, x_2, ..., x_n \right]$. The distance between each $x_i$ variable is controlled by the step size, $h = \frac{x_n -x_0}{n}$, this gives an expression for $x_i = x_0 + i\cdot h$ where $i = 1,2,3, ..., n$. 
\par A widely used method to calculate the derivate numerically is what is called the 3 point method equation (\ref{eq:2}), where $f_{i \pm 1}$ is a shorthand for $ f(x_i \pm h)$.   

\begin{equation}\label{eq:2}
  f'_i = \frac{f_{i+1} - f_{i-1}}{2h}
\end{equation}
The three point formula has an error of order of magnitude $O(h^2)$ compared to a two point method which has $O(h)$, while requiring the same number of floating point operations (flops). The idea behind the three point method is that you eva    
\newpage
\printbibliography

\end{document}