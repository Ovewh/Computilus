\section{Methods}


\subsection{Ising model}

The Ising model, without an external magnetic field, is given by
\cref{eq:ising}, where s$_k$ and s$_l$ takes values of $\pm 1$ and N is the
number of spins. We will use quadratic grids with grid width L and periodic
boundary conditions.

\begin{equation}
  \label{eq:ising}
  E = -J\sum_{<kl>}^{L^2} s_k s_l
\end{equation}


\subsection{Metropolis algorithm}

Since the Ising model is a binary system the number of possible microstates is
2$^{L^2}$. Even for small systems it is therefore computationally expensive to use
the brute force method of generating all possible states and calculating
expectation values. In the next section we do this for L=2.

\subsection{Analytic solutions}

For a 2x2 grid it is doable to find analytical solution for the values of
interest that we can use to compare the results of the numerical approximations.
We start by calculating the energy for a given configuration.



\begin{align*}
  E_i &= -J \sum_{<k,l>} s_k s_l = -\frac{1}{2}J \sum_{i} \sum_{nn} s_i s_{nn} \\
      &= -\frac{1}{2} J \brak{2 s_1 (s_2 + s_3) + 2s_2(s_1 + s_4) + 2s_3(s_4 + s_1) + 2s_4 (s_2 + s_3)}\\
      &= -J\brak{ 2(s_4 + s_1)(s_2 + s_3)} \\
      &= -2J\brak{(s_4 + s_1)(s_2+s_3)}
\end{align*}

To calculate the expectation values we need to know the possible energies and
magnetizations. The script analytic.py \parencite{github} generates all the
possible states and lists the values we need as shown in \cref{tab:analytic}.

\begin{table}[htp]
  \centering
  \csvautotabular{../data/analytic.csv}
  \caption{Analytical values for a 2x2 grid.}
  \label{tab:analytic}
\end{table}

The probability of a given configuration is given by \cref{eq:prob}, with
the z being a normalization factor to ensure the sum of probabilities are one,
known as the partition function, given in \cref{eq:partition}.


\begin{equation}
  \label{eq:prob}
  P_i = \frac{\exp(-\beta E_i)}{z}
\end{equation}

\begin{equation}
  \label{eq:partition}
  z = \sum_{i}^{N} \exp(-\beta E_i) = \exp(8\beta) + 12 + 2\exp(-8\beta)
\end{equation}

\subsection{Model testing}

Implementation of tests (test\_analytic.py, test\_ising.py) showed that the
analytical solutions was correct down to an absolute error of \num{1e-12}  when
comparing against a brute force method where we created all the possible
microstates.
Testing the ising model against the analytical solutions for a 2x2 grid showed
that we could consistently achieve a relative error of 0.001 with only 100000
Monte Carlo cycles (MCC) for the energy and magnetization. For the heat capacity
and susceptibility we had to increase the MCC to 5 millions and the relative
error tolerance to 0.05.


\subsection{Production runs}

For the production runs (phase\_transitions.py) we parallelized the code with
numba using $@njit(parallel=True)$. We found it easiest to run calculations for
each temperature in parallel, instead of parallelizing the ising model itself.
An unfortunate side effect of this was that we were not able to write to file
after each run of the ising model, so if our run did not finish for some reason
we would lose all the results.
We also added a script (timings.py) that estimates the time to run phase\_transitions.py
with the given parameters so we did not start too long runs.

Timings with and without parallelization for small grid sizes showed a speedup
of roughly 3.4 on a laptop with 4 physical cores. The suboptimal speedup might
be related to creation of and writing to the 2d arrays for storing results.
