\section{Methods}


\subsection{Analytic solutions}

For a 2x2 grid it is doable to find analytical solution for the values of
interest that we can use to compare the results of the numerical approximations.
We start by calculating the energy for a given configuration.

\begin{align*}
  E_i &= -J \sum_{<k,l>} s_k s_l = -\frac{1}{2}J \sum_{i} \sum_{nn} s_i s_{nn} \\
      &= -\frac{1}{2} J \brak{2 s_1 (s_2 + s_3) + 2s_2(s_1 + s_4) + 2s_3(s_4 + s_1) + 2s_4 (s_2 + s_3)}\\
      &= -J\brak{ 2(s_4 + s_1)(s_2 + s_3)} \\
      &= -2J\brak{(s_4 + s_1)(s_2+s_3)}
\end{align*}

To calculate the expectation values we need to know the possible energies and
magnetizations. The script analytic.py \parencite{github} generates all the
possible states and lists the values we need as shown in \cref{tab:analytic}.

\begin{table}[htp]
  \centering
  \csvautotabular{../data/analytic.csv}
  \caption{Analytical values for a 2x2 grid.}
  \label{tab:analytic}
\end{table}

The probability of a given configuration is given by \cref{eq:prob}, with
the z being a normalization factor to ensure the sum of probabilities are one,
known as the partition function, given in \cref{eq:partition}.


\begin{equation}
  \label{eq:prob}
  P_i = \frac{\exp(-\beta E_i)}{z}
\end{equation}

\begin{equation}
  \label{eq:partition}
  z = \sum_{i}^{N} \exp(-\beta E_i) = \exp(8\beta) + 12 + 2\exp(-8\beta)
\end{equation}

\subsection{Model testing}

Implementation of tests (test\_analytic.py, test\_ising.py) showed that the
analytical solutions was correct down to an absolute error of \num{1e-12}  when
comparing against a brute force method where we created all the possible
microstates.
Testing the ising model against the analytical solutions for a 2x2 grid showed
that we could consistently achieve a relative error of 0.001 with only 100000
Monte Carlo cycles (MCC) for the energy and magnetization. For the heat capacity
and susceptibility we had to increase the MCC to 5 millions and the relative
error tolerance to 0.05.
