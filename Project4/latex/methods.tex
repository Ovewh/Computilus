\section{Methods}


\subsection{Analytic solutions}

For a 2x2 grid it is doable to find analytical solution for the values of
interest that we can use to compare results for the numerical solutions.
We start by calculating the energy for a given configuration.

\begin{align*}
  E_i &= -J \sum_{<k,l>} s_k s_l = -\frac{1}{2}J \sum_{i} \sum_{n,n} s_i s_{nn} \\
      &= -\frac{1}{2} J \brak{2 s_1 (s_2 + s_3) + 2s_2(s_1 + s_4) + 2s_3(s_4 + s_1) + 2s_4 (s_2 + s_3)}\\
      &= -J\brak{ 2(s_4 + s_1)(s_2 + s_3)} \\
      &= -2J\brak{(s_4 + s_1)(s_2+s_3)}
\end{align*}

To calculate the expectation values we need to know the possible energies and
magnetizations. The script analytic.py \parencite{github} generates all the
possible states and lists the values we need as shown in \cref{tab:analytic}.

\begin{table}[htp]
  \centering
  \csvautotabular{../data/analytic.csv}
  \caption{Analytical values for a 2x2 grid.}
  \label{tab:analytic}
\end{table}

The probability of a given configuration is given by \cref{eq:prob}, with
the z being a normalization factor to ensure the sum of probabilities are one,
known as the partition function, given in \cref{eq:partition}.


\begin{equation}
  P_i = \frac{\exp(-\beta E_i)}{z}
  \label{eq:prob}
\end{equation}

\begin{equation}
  \label{eq:partition}
  z = \sum_{i}^{N} \exp(-\beta E_i) = \exp(8\beta) + 12 + 2\exp(-8\beta)
\end{equation}





Present results of following calculations, but move to appendix.


\begin{align}
  \braket{E} &= \sum_{i=1}^{N} E_i P_i \\
             &= \brak{2 \cdot -8 \exp(8\beta) + 2\cdot 8 \exp(-8\beta)} / z \\
             &= \frac{16}{z} ( \exp(-8\beta) - \exp(8\beta))
\end{align}


\begin{align}
  \braket{E^2} &= \sum_{i=1}^{N} E_i^2 P_i \\
                &= \brak{ 2 (-8\exp(8\beta))^2 + 2 (8\exp(-8\beta))^2 } /z \\
                &= \frac{128}{z}\brak{ \exp(16 \beta) + \exp(-16\beta)}
\end{align}

\begin{align}
  \braket{E}^2 &= \frac{16}{z}^2 ( 2 + \exp(-16\beta) + \exp{16\beta})
\end{align}

\begin{align}
  \braket{M} &= \sum_{i=1}^{N} \abs{M_i} P_i \\
             &= \brak{2\cdot 4 \exp(-8\beta ) + 8 \cdot 2 \exp(0)} / z \\
             &= \frac{8}{z} \para{ 1 - 2\exp(-8\beta) }
\end{align}


\begin{align}
  \braket{M^2} &= \sum_{i=1}^{N} \abs{M_i}^2 P_i \\
               &= \brak{2 (4\exp(-8\beta))^2 + 8 (2)^2} / z \\
               &= \frac{32}{z} \brak{ \exp(-16\beta) + 1 }
\end{align}

\begin{align}
  \braket{M}^2 &= \frac{8}{z}^2 \para{ 1 - 2\exp(-8\beta) }^2 \\
               &= \frac{64}{z^2} \para{ 1 - 4\exp(-8\beta) + 4\exp(-16\beta) }
\end{align}


\begin{align}
  \sigma_{E}^2 &= \braket{E^2} - \braket{E}^2 \\
               &= \frac{128}{z}\brak{ \exp(16 \beta) + \exp(-16\beta)} - \frac{16}{z}^2 ( 2 + \exp(-16\beta) + \exp(16\beta))
\end{align}



\begin{align}
  \sigma_{M}^2 &= \braket{M^2} - \braket{M}^2 \\
               &=  \frac{32}{z} \brak{ \exp(-16\beta) + 1 } - \frac{64}{z^2} \para{ 1 - 4\exp(-8\beta) + 4\exp(-16\beta) }
\end{align}
