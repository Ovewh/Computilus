\newpage

\section{Discussion and Conclusion}

Our model seems to have captured the phase transition well. As we increased the
grid size the phase change in magnetization started to behave more and more like
a discontinuity.
Compared to Onsager's results (\cref{eq:analytical}) we got an absolute error
of \num{4.54e-04} and a relative error of \num{2.00e-4} which we are very
pleased with.

The values for the confidence intervals around the T$_c$(L=$\infty$) assumes
that the values for T$_c$(L) are certain. This is not correct, since we have
uncertainties inherent in the Monte Carlo simulations and in the method we use
to estimate T$_c$(L). As shown in \cref{fig:error_L2} our model exhibited
relative errors on the order of \num{1e-3} compared to the analytic solution
using a 2x2 grid. If the errors on a larger grid are comparable this should not
be a major source of error.

We tested two other methods for estimating T$_C(L)$. The first method was to
take the mean of the n points with largest heat capacity. The second was to take
a rolling mean with a window of n. We did this for various n, and as long as we
kept n reasonably small compared to the 300 different temperatures of our
dataset this gave similar results for T$_C(L)$ compared to fitting a sixth order
polynomial. Therefore the method we have used for selecting T$_C(L)$ does not
seem to be a major source of error either.




We encountered a few problems using numba to speed up and parallelize the code.
The first was that it is generally difficult to split up code into smaller
functions that can be unit tested. The other was that we did not manage to both
parallelize and write to file after the run for each temperature and grid size.
This meant we had to do the production runs, that took roughly 15 hours,
without the possibility of looking at the current progress. In one run this
led to no results being written to file, since there was an error in the file
writing function.
