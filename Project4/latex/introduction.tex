\section{Introduction}
There are many phenomenons that arise as a result of physical entities
interacting with each other and together producing a global behaviour.
Ferromagnetism is the result of such collective behaviour. A widely used model
for studying these collective phenomenons is the Ising model.
The model was first suggested by Wilhelm Lenz in 1920 for modelling
ferromagnetism and later solved 
in one dimension by his PhD student Erst Ising in 1925. Since the nineteen
twenties the Ising model risen to become one of the most studied models in
statistical mechanics and has even
transcended the traditional boundaries between the disciplines. Finding
application ranging from social sciences and biology to chemistry and
mathematics.

In this work we will look at a classic application of the Ising model, namely
in studying phase transitions in magnetic systems. We will use a common
stochastic Monte Carlo model of the Ising model, built around the Metropolis algorithm. We will give a through explanation of our numerical model
and demonstrate the behaviour of the model at different temperatures, lattice
sizes and initial states. The aim with our analysis is to determine the
temperature where the magnetic system transitions from a 
magnetized state to a non-magnetized state. This
temperature will subsequently be refereed to as the critical
temperature of the system. We will find the critical temperature for several lattice sizes
and then extrapolate the critical temperatures to in order to estimate the
critical temperature for an infinite lattice and
check our estimate against Lars Onsager's analytical solution
\cref{eq:analytical} \parencite{Lars1944}.

\begin{equation}
  \label{eq:analytical}
  T_C(L=\infty) = \frac{2}{\ln(1 + \sqrt2)} \approx 2.269
\end{equation}


%https://arxiv.org/pdf/1706.01764.pdf

%https://iopscience.iop.org/article/10.1088/0143-0807/37/6/065103
