\section{Introduction}
There are many phenomenons that arise as a result of many physical entities
interacting with each other to produce some global behaviour.
Ferromagnetism could be an example of such a phenomenon. A widely used model for
studying such phenomenons is the Ising model.
The model was first suggested by Wilhelm Lenz in 1920 for modelling ferromagnetism and later solved
for one dimension by his PhD student Erst Ising in 1925. Since the nineteen
twenties the Ising model risen to become one of the most used models in
statistical mechanics and has even
transcended the traditional boundaries between the disciplines. Finding
application ranging from social sciences and biology to chemistry and
mathematics.

In this work we will look at a classic application of the Ising model, namely
the phase transitions in magnetic systems.
We will explain the theory of the ising model and the metropolis algorithm,
before deriving analytical solutions for expectation values of the ising
model using a 2x2 grid. These results will be used as a benchmark for testing
our implementation. After investigating when the most likely state is reached
for a 20x20 lattice we will do runs with grid sizes of 40, 60, 80 and 100
for different temperatures to look for a phase transitions.
We will then find the critical temperature for each grid size and extrapolate
this to find the critical temperature in the limit of an infinite grid.

%https://arxiv.org/pdf/1706.01764.pdf

%https://iopscience.iop.org/article/10.1088/0143-0807/37/6/065103
