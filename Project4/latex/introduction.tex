\section{Introduction}
There are many phenomenons that arise as a result of many physical entities
interacting with each other and producing some global behaviour,
ferromagnetism is an example of such a phenomenon. A widely used model for studying
such phenomena is the Ising model.
The model was first suggested by Wilhelm Lenz in 1920 for modelling ferromagnetism and later solved
for one dimension by his PhD student Erst Ising in 1925. Since the nineteen
twenties the Ising model risen to become one of the most used models in
statistical mechanics and has even
transcended the traditional boundaries between the disciplines. Finding 
application ranging from social sciences and biology to chemistry and
mathematics.     

In this work we will apply the Ising model in studying phase
transitions in magnetic systems, the phenomenon that takes us
back to the Ising model's origin.  This is conceptually very simple systems,
consisting of binary spins which can either have the value 1 (up) or -1 (down).
The energy of this system is given by summing up nearest neighbour spins of all
spins in our system. We evolve our system by randomly selecting spins and
changing their values if it is energetically favourable for the system. To not
have our system being stuck in the ground state (lowest energy state) we will randomly flip some spins
anyway even if it is not energetically favourable, this behaviour is actually
something for real systems to. 

%https://arxiv.org/pdf/1706.01764.pdf

%https://iopscience.iop.org/article/10.1088/0143-0807/37/6/065103